\documentclass{article}

\usepackage{chinese}
\usepackage{listings}
%\usepackage{varioref}

\lstset{frame=tb,
    language=Python,
}

\title{採用爬山演算法自動建構神經網路模型 -- 以 MNIST 為例}

\author{
    陳鍾誠 \\
    國立金門大學 資訊工程學系\\
    \texttt{ccc@nqu.edu.tw} \\
}

\begin{document}
\maketitle

\begin{abstract}
目前的神經網路或深度學習模型,通常是由研究者根據經驗與直覺建構出來的,然後再透過實驗檢驗其模型的好壞。

但若能用程式自動建構神經網路的架構,除了不需要依賴人腦的直覺之外,還有可能建構出人腦所難以想出來的模型。

本論文針對手寫數字辨識問題,在 MNIST 資料集上,採用爬山演算法進行了初步的《自動建構神經網路》實驗!
\\
\\
本論文的開放原始碼專案網址為:\url{ https://github.com/cccresearch/nnModelAuto/ }
\end{abstract}


% keywords can be removed
\keywords{神經網路\and 深度學習\and MNIST\and 自動建模}


\section{簡介}

深度學習的神經網路的學習演算法,目前仍然以梯度下降法為主流,透過反傳遞的方式自動微分,計算出梯度,然後向梯度方向邁出微小的步伐,直到無法繼續下降為止。

但是梯度下降法只適用於《連續可微分函數》,對於那些難以轉換成《連續可微分函數》的問題而言,通常無法直接使用梯度下降法。

神經網路的模型,目前通常是由《數十種類型的網路層》所疊加起來的,像是 Linear/ReLU/Conv2D/Pool/Sigmoid 等等。我們通常很難將模型化為某種《連續可微分函數》,因此難以使用梯度下降法來自動調整網路架構。

不過若採用傳統人工智慧中的《搜尋法或優化算法》,則不需要透過梯度下降法。例如我們認為可以用《爬山演算法》或《機率式的優先搜尋法》,進行神經網路架構調整,自動找出好的神經網路模型,似乎是一種可行的想法!

為了驗證這樣的想法,我們嘗試使用傳統人工智慧中簡單的《爬山演算法》,針對手寫辨識 MNIST 測試集,自動建構神經網路模型。

我們的實驗結果顯示,使用簡單的爬山演算法,就能從一個單層線性網路的架構開始,在正確率上逐步攀爬,最後得到一個還不錯的多層架構,讓正確率從 91.79\% 提升到 98.05\%。

當然,若採用其他的優化方法,例如《最佳優先搜尋法》等,或許會比爬山演算法表現更好,因此本實驗只能算是利用優化方法自動建模的一個初步嘗試。

\section{背景}

神經網路的研究從 Frank Rosenblatt 在 1957 年發明的《單層感知器》開始萌芽 \cite{Rosenblatt1958ThePA} ,到了 Hinton 等人在 1986 年 \cite{Rumelhart1986LearningRB} 重新發明反傳遞演算法,並成功應用在語音辨識等問題之後,有過一陣研究熱潮。

熱潮退去後,仍有些研究者持續改良神經網路模型,像是 LeCun \cite{LeCun1998GradientbasedLA} 等人所提出的卷積神經網路,在影像識別領域就有優異的表現。

2011 年繪圖處理器硬體上的進展,以及 ImageNet 等大量測試集的出現,引發了神經網路的新一波研究發展,很多論文改良了卷積神經網路的架構,讓神經網路的層數加深並在許多領域表現優異,這些新發產被統稱為《深度學習技術》 \cite{Szegedy2015GoingDW} \cite{Ioffe2015BatchNA} \cite{Chollet2017XceptionDL} 。

然而這些神經網路架構,通常是由創造者的知識經驗所設計出來的。

於是我們不禁想問,既然神經網路可以建構出《影像辨識、語音辨識、語言生成》等模型,那麼我們能否用程式自動建構出神經網路呢?

目前、這類的研究並不多,但已經逐漸有研究者投入 \cite{Mendoza2019TowardsAD} \cite{Abreu2019AutoNN},本研究也是我們的一個初步嘗試。

\section{方法}

爬山演算法是模仿爬山的動作,只要看到附近有更高的點,就往那個方向爬,寫成演算法如下所示:

\begin{minipage}{\linewidth}
\begin{verbatim}
Algorithm HillClimbing(f, x)
    x = 隨意設定一個解。
    while (x 有鄰居 x' 比 x 更高)
        x = x';
    end
    return x;
end
\end{verbatim}
\end{minipage}

由於要比較高低,因此通常會設定固定的高度函數 height(),透過 height() 去比較兩個解答的高度,然後決定新解達是否比舊解答的高度更高;若新解答更高則移動過去,否則就繼續找下一個鄰居。

如果嘗試了很多次,都找不到更高的鄰居,那就認為已經爬到某個山頂,也就是區域最佳解,於是爬山演算法就會結束離開。

以下 Python 程式是上述演算法的更詳細版本,也是本文實驗所採用的方法!

\begin{minipage}{\linewidth}
\begin{lstlisting}
def hillClimbing(s, maxGens, maxFails):   # 爬山演算法的主體函數
    global file
    file = open('./model/hillClimbing.log', 'w')
    log(f"start: {str(s)}")               # 印出初始解
    fails = 0                             # 失敗次數設為 0
    # 當代數 gen<maxGen,且連續失敗次數 fails < maxFails 時,就持續嘗試尋找更好的解。
    for gens in range(maxGens):
        snew = s.neighbor()               #  取得鄰近的解
        # log(f'snew={str(snew)}')
        sheight = s.height()              #  sheight=目前解的高度
        nheight = snew.height()           #  nheight=鄰近解的高度
        # log(f'sheight:{sheight} nheight:{nheight}')
        if (nheight > sheight):           #  如果鄰近解比目前解更好
            log(f'{gens}:{str(snew)}')    #    印出新的解
            s = snew                      #    就移動過去
            fails = 0                     #    移動成功,將連續失敗次數歸零
        else:                             #  否則
            fails = fails + 1             #    將連續失敗次數加一
        if (fails >= maxFails):
            log(f'fail {fails} times!')
            break
    log(f"solution: {str(s)}")            #  印出最後找到的那個解
    file.close()
    return s                              #    然後傳回。
\end{lstlisting}
\end{minipage}

上述演算法有兩個重要的函數未交代清楚,一個是 height() ,另一個是 neighbor() 。

在我們的實驗中,採用《正確率 - 神經網路複雜度》作為高度的衡量。其中的神經網路複雜度設定為《網路的參數數量/一百萬》,對應的 Python 程式碼如下。

\begin{minipage}{\linewidth}
\begin{lstlisting}
def height(self):
    net = self.net
    if not net.exist():        # 如果之前沒訓練過這個模型
        trainer.run(net)       # 那麼就先訓練並紀錄正確率
    else:
        net.load()             # 載入先前紀錄的模型與正確率
    # 傳回高度 = 正確率 - 網路的參數數量/一百萬
    return net.accuracy()-(net.parameter_count()/1000000)
\end{lstlisting}
\end{minipage}

這樣的高度設計並非是最好的,而且有人為調整的空間,目前採用這個公式只是個初步嘗試。

鄰居函數 neighbor() 的設計,則是採用《隨機選取操作》的方式,可用的操作有《新增與修改》,其中新增是增加一個神經網路層,而修改則是將一個網路層取代為另一個隨機產生的網路層。

\begin{minipage}{\linewidth}
\begin{lstlisting}
def neighbor(self):
    model = copy.deepcopy(self.net.model)    # 複製模型
    layers = model["layers"]                 # 取得網路層次
    in_shapes = self.net.in_shapes           # 取得各層次的輸入形狀
    ops = ["insert", "update"]               # 可用的操作有新增和修改
    success = False
    while not success:                       # 直到成功產生一個合格鄰居為止
        i = random.randint(0, len(layers)-1) # 隨機選取第 i 層 (進行修改或新增)
        layer = layers[i]
        op = random.choice(ops)              # 隨機選取操作 (修改或新增)
        newLayer = randomLayer()             # 隨機產生一個網路層
        if not compatable(in_shapes[i], newLayer["type"]): # 若新層不相容 (輸入維度不對)
            continue                         #   那麼就重新產生
        if op == "insert":                   # 如果是新增操作
            layers.insert(i, newLayer)       #   就插入到第 i 層之後
        elif op == "update":                 # 如果是修改操作
            if layers[i]["type"] == "Flatten": # 不能把 Flatten 層改掉
                continue                       # (因為我們強制只能有一個 Flatten 層)
            else:
                layers[i] = newLayer         # 若不是 Flatten 層則可以修改之
        break

    nNet = Net()                             # 創建新網路物件
    nNet.build(model)                        # 根據調整後的 model 建立神經網路
    return SolutionNet(nNet)                 # 傳回新建立的爬山演算法解答
\end{lstlisting}
\end{minipage}

由於神經網路的訓練相當耗時,因此在隨機產生網路層時,我們限縮了鄰居的可能性,以避免產生過多的可能鄰居,導致速度太慢,目前只有下列程式中 types 所指定的六種網路層可以選取。

基於同樣的理由,對於網路層的參數,像是大小與通道數也都不能任意選,基本上都是以 2 的次方為選擇項,這樣才不會有太多的鄰居。

\begin{minipage}{\linewidth}
\begin{lstlisting}
types = ["ReLU", "Linear", "Conv2d", "AvgPool2d", "LinearReLU", "ConvPool2d"]
sizes = [ 8, 16, 32, 64, 128, 256 ] # 限縮大小選取範圍,不是所有整數都可以
channels = [ 1, 2, 4, 8, 16, 32 ]   # 限縮通道數範圍

def randomLayer():
    type1 = random.choice(types)             # 隨機選一種層次
    if type1 in ["Linear", "LinearReLU"]:    # 如果是 Linear 或 LinearReLU
        k = random.choice(sizes)             #   就隨機選取 k 作為輸出節點數
        return {"type":type1, "整里程表tures":k}
    elif type1 in ["Conv2d", "ConvPool2d"]:  # 如果是 Conv2d 或 ConvPool2d
        out_channels = random.choice(channels) # 就隨機選取 channels 數量
        return {"type":type1, "out_channels": out_channels}
    else:                                    
        return {"type":type1}                # 否則不須設定參數,直接傳回該隨機層。
\end{lstlisting}
\end{minipage}

必須小心的是,並不是所有隨機產生的層都可以任意插入,因此必須先檢查相容性 (輸入維度是否正確) 後才能進行《新增與修改》動作,以下是相容性檢查的算法。

\begin{minipage}{\linewidth}
\begin{lstlisting}
types2d = ["Conv2d", "ConvPool2d", "AvgPool2d", "Flatten"]
types1d = ["Linear", "LinearReLU"]

def compatable(in_shape, newLayerType):
    if newLayerType in ["ReLU"]: # 任何維度都可以使用 ReLU 操作
        return True
    elif len(in_shape) == 4 and newLayerType in types2d:
        # 這些層的輸入必須是 4 維的 (1. 樣本數 2. 通道數 3. 寬 4. 高)
        return True # 
    elif len(in_shape) == 2 and newLayerType in types1d:
        # 這些層的輸入必須是 2 維 (1. 樣本數 2. 輸出節點數)
        return True
    return False
\end{lstlisting}
\end{minipage}

最後,爬山演算法必須有個起點,我們選擇用單層網路作為起點,也就是只有《Flatten + Linear》所形成的網路,其中 Flatten 是為了將原本 MNIST 輸入的多維《通道+影像》結構,攤平成單一維度的結構 (在 PyTorch 中得加上樣本數這個維度,所以是二維結構),而 Linear 是線性全連接層,將攤平後的輸入直接接到輸出層。

從這樣的起點出發,透過爬山演算法,逐步加入新的網路層,或者修改某層,只要找到更高的層,就接受這個更好的網路,讓爬山演算法爬過去那裏。

\section{實驗結果}

本實驗中的爬山演算法,會在發現更好的模型時,印出該模型,以下是某次實驗整理成表格後的結果:

\begin{table}
 \caption{不同模型的 MNIST 正確率}
  \centering
  \begin{tabular}{llllll}
    編號 & 模型     & 正確率     & 參數數量 & 高度    & 說明 \\
    \midrule
    0  & Flatten & 91.79\% & 7850 & 91.782151 & 起點\\
    1   & ReLU+Flatten & 91.91\% & 7850 & 91.902154  \\
    5   & ConvPool2d(8)+Flatten & 92.08\% & 13610 & 92.066392 \\
    9   & ConvPool2d(8)+ReLU+Flatten & 93.37\% & 13610 & 93.356393      \\
    40  & Conv2d(4)+ReLU+Flatten & 95.98\% & 27090 & 95.952913       \\
    47  & Conv2d(4)+Conv2d(8)+ReLU+Flatten & 97.64\% & 46426 & 97.593573       \\
    49  & Conv2d(4)+Conv2d(8)+Conv2d(32)+ReLU+Flatten & 98.05\% & 157562 & 97.892441 \\
    \bottomrule
  \end{tabular}
  \label{table:experiment1}
\end{table}

從表格 \ref{table:experiment1} 中,您可以看到爬山演算法從初始模型開始,將正確率從 91.79\% 開始逐步提升,每次只會新增或修改一層,最後建構出了 Conv2d(4)+Conv2d(8)+Conv2d(32)+ReLU+Flatten 這個模型,其正確率為 98.05\%。

但是模型的參數數量,也從 7850 開始,一路提高到 157562,但套上我們的高度公式《高度 = 正確率 - 參數數量/一百萬》,其中模型複雜度《參數數量/一百萬》=《157562/1000000》 大約只扣了 0.158 分,因此該模型分數為 98.05-0.158 = 97.892,仍然勝過其他模型。

我們可以透過爬山演算法的高度函數設計,控制參數數量的多寡,例如我們若偏好小型模型,那麼或許可以將高度公式改為 《高度 = 正確率 - 參數數量/十萬》,但若我們認為模型大無所謂,那麼使用《高度 = 正確率》 這樣的公式也就可以了。

\section{結論與展望}

以上的實驗,證實了我們可以透過《爬山演算法自動建立神經網路模型》的想法,確實是可行的。

當然,爬山演算法並非唯一可用的方法,若能使用《機率式的優化搜尋法》,或許會比爬山演算法取得更好的結果,這有待後續的實驗去驗證或否證。

另外、由於我們的實驗電腦沒有使用 GPU,因此執行速度受限,目前僅能在 MNIST 這樣的小問題上進行實驗。若能取得具備 GPU 的電腦,就有機會將這樣的方法擴充到 CIFAR 或 ImageNet 等較複雜的大規模測試集上進行實驗,以進一步觀察自動建模的能力是否夠好,以及是否需要修改或加入更多的網路層,讓系統能克服這些更複雜的問題。

在本論文中,我們目前只鎖定影像辨識領域,但我們認為在《語言生成》等適合《循環神經網路》模型去學習的領域,或許也能用類似的方式,自動建構出適當的《循環神經網路模型》,這也仍然是有待實驗去證實或否證的問題。



\renewcommand\refname{參考文獻}

\bibliographystyle{unsrt}  
\bibliography{references}  %%% Remove comment to use the external .bib file (using bibtex).
%%% and comment out the ``thebibliography'' section.


% \end{thebibliography}


\end{document}
